\documentclass[a4paper,11pt]{article}
\usepackage[T1]{fontenc}
\usepackage[utf8]{inputenc}
\usepackage{lmodern}
\usepackage[brazil]{babel}
\usepackage{listings}
\lstset{language=C}
\usepackage{tikz}
\usepackage{graphicx}
\usepackage{amsmath}
\usepackage{amsthm}
\usepackage{amsfonts}
\PassOptionsToPackage{usenames,dvipsnames,svgnames}{xcolor}  
\usetikzlibrary{arrows,positioning,automata}
\begin{document}

\begin{center}{\Large \bf Documenta\c{c}ão EP2 Canoagem \\ }\end{center}
\begin{center}{\Large \bf Manual do Desenvolvedor\\ }\end{center}

\begin{center}
{
Fellipe Souto Sampaio\footnote{e-mail: fellipe.sampaio@usp.com}
Gervásio Santos \footnote{foo}
Vinícius Verdamini \footnote{foo2}
}

\end{center}

\begin{center}
MAC 0211 Laboratório de Programa\c{c}ão I \\
Prof. Kelly Rosa Braghetto \\
             
\end{center}

\begin{center}
Instituto de Matemática e Estatística - IME USP \\
 Rua do Matão 1010 \\
 05311-970\, Cidade Universitária, São Paulo - SP \\
\end{center}

\newpage

\section{Introdu\c{c}ão}
Esta documenta\c{c}ão apresenta uma completa descri\c{c}ão do funcionamento do exercício programa 2-A e seus algoritmos, divido através de sua implementa\c{c}ão nos .c e .h .

\section{Pixel}
Para modelar o funcionamento da grade decidiu-se implementar um matriz de tipo pixel. Nesta estrura existe dois campos, \textit{velocidade} e \textit{tipo} que guardam a velocidade e o tipo do ponto respectivamente. Para lidar com estar estrutura foi implemetado as seguinte fun\c{c}ões:
\begin{enumerate}
\item[•]{velocidade - Retorna a velocidade de um ponto.}
\item[•]{tipo - Retorna o tipo de um ponto.} 
\item[•]{setaVelocidade - Muda a velocidade do ponto para o valor fornecido.} 
\item[•]{setaTipo - Muda a natureza do ponto para o tipo fornecido.} 

\end{enumerate}


\section{Rio}
Esta interface lida com toda opera\c{c}ão dos pontos, desde a gera\c{c}ão das margens, ilhas, velocidade da água entre outras. Temos duas fun\c{c} principais, primeiraLinha e proximaLinha, as quais estão disponíveis para outras interfaces. Seus métodos internos são chamados para constru\c{c}ão das linhas, cada qual exercendo uma funcionalidade própria. O fluxo de funcionamento das duas fun\c{c} está descrito nos anexos 1 e 2 respectivamente, onde cada vértice representa um método aplicado, o valor das aretas é a ordem em que são chamadas e por fim a orienta\c{c}ão do grafo informa o fluxo de informa\c{c}ões trocadas entre a fun\c{c} chamadora e a chamada.\\

\subsection{primeiraLinha}
Esta fun\c{c}ão tem a funcionalidade de criar a primeira linha da grade, que servirá como linha base para linhas subsequentes. Chama-se o método \textit{tamanhoDaPrimeiraMargem} para definir o tamaho das margens esquerda e direita, este método trabalha com a macro limiteDasMargens definida em main.c . Após utilizar os métodos descritos em pixel para organizar o tipo do ponto e sua velocidade chama-se \textit{velocidadeDaAguaPrimeiraLinha**} que atribui uma velocidade randômica dentro do intervalo $[-0.5,0.5]$ para o primeiro ponto com água fora da margem, para os pontos seguintes repete-se o passo da forma descrita na equa\c{c}ão a seguir:

\begin{equation}
v_i = v_{i-1} + \lambda \textbf{\, , } \lambda \in [-0.5,0.5]
\end{equation}

Após inserir as devidas velocidades o método \textit{insereIlha} tem o trabalho de inserir um obstáculo na por\c{c}ão central do rio de acordo com a proba-bilidade inserida pelo usuário. Um valor default definido em main.c é atribuido a probabilidade a distância mínima entre ilhas caso o usuário não fa\c{c}a a inser\c{c}ão destes valores. Inteiros aleatórios são gerados para simular a probabilidade P de existir ilhas ou não, seu tamanho está dentro dos limites da terceira por\c{c}ão da grade. Com intuito de evitar o encontro de uma das margens com uma dada ilha é considerado que \textbf{a distância mínima entre os dois deve ser de 2 pontos}. Ao fim de todos estes passos, de gera\c{c}ão à inser\c{c}ão, é necessário ajustar o fluxo da água na linha. Primeiro suave-se a velocidade dos pontos	 para que a diferen\c{c}a entre eles não seja tão discrepante, isto é feito em \textit{suavizaVelocidade}, e então a velocidade dos pontos é normalizada com o uso da formula de interpola\c{c}ão: 

\begin{equation}
\phi = \sum_{i=0}^{N} v_i
\end{equation}
\begin{equation}
v_i \leftarrow v_i\cdot\dfrac{\Phi}{\phi}
\end{equation}
\\
onde $\Phi$ é o fluxo inserido pelo usuário.

\subsection{proximaLinha}

A gera\c{c}ão das linhas subsequentes funciona de modo semelhante a fun\c{c}ão primeiraLinha. Cada nova linha utiliza a anterior como base para sua cria\c{c}ão. As diferen\c{c}as residem nos métodos que lidam com as margens e com a velocidade das linhas. O método \textit{aleatorizaMargem} analisa as margens anteriores e sorteia um valor inteiro no intervalo $[-1,1]$. Este valor é somado as margens da linha anterior e atribuindo-se à nova, respeitando sempre o limite das margens, isso garante que a mudan\c{c}a entre a margem anterior e atual seja suave. Para gera\c{c}ão de velocidades na nova linha o método \textit{velocidadeProximaLinha} é empregado. Com intuito de preservar a suavidade da velocidade entre pontos é tomado como base a velocidade do ponto anterior e realizado o produto desta com um valor aleatório real no interválo $[0.9,1.1]$, e no caso do ponto anterior ter velocidade zero, realiza-se um outro cálculo.

\section{Grade}
Esta interface concentra o método de cria\c{c}ão da matriz e dos frames utilizados durante o programa. O método \textit{initGrade} aloca dinamicamente uma matriz de tamanho $altura \times largura$, as quais tem valor default de 30 por 100 respetivamente. Em \textit{criaPrimeiroFrame} cria-se a primeira imagem que será impressa, a constru\c{c}ão das linhas acontece de baixo para cima, indo da ultima linha da matriz até a primeira. A mesma ideia é utilizada em \textit{criaProximoFrame}, entretanto nesta é criada apenas uma nova linha e inserida na posi\c{c}ão $(indice + alturaDaGrade -1)\text{ \, } mod \text{ \, } alturaDaGrade$ . Um método de desaloca\c{c}ão de memória, \textit{freeGrade}, foi implementado para situa\c{c}ões futuras nas quais a execu\c{c}ão do programa não seja interrompida pelo comando CTRL+C.














\newpage
\begin{center}
\textbf{\text{\Large  Anexo 1 - Fluxo de funcionamento primeiraLinha}}
\begin{tikzpicture}[>=stealth',shorten >=1pt,node distance=3cm,on grid,initial/.style    ={}]
  \node[state]          (P)                        {$pL$};
  \node[state]          (B) [ above =of P]    {$tPM$};
  \node[state]          (M) [above right =of P]    {$velAPL$};
  \node[state]          (D) [right =of P]    {$iIlha	$};
  \node[state]          (C) [below right =of P]    {$nLz$};
\tikzset{mystyle/.style={<->,double=orange}} 
\tikzset{every node/.style={fill=white}} 
\path (P)     edge [mystyle]    node   {$1$} (B)
      (P)     edge [mystyle]    node   {$2$} (M) 
      (P)     edge [mystyle]    node   {$3$} (D)
      (P)     edge [mystyle]    node   {$4$} (C);

\textbf{\text{ primeiraLinha - Legenda }}
\end{tikzpicture}
\end{center}
\textbf{\text{  primeimaLinha - Legenda }}
\begin{enumerate}
\item[-]{pL - primeiraLinha}
\item[-]{tPM - tamanhoDaPrimeiraMargem}
\item[-]{velAPL - velocidadeDaAguaPrimeiraLinha}
\item[-]{iIlha - InsereIlha}
\item[-]{nLz - normaliza}
\end{enumerate}


\newpage
\begin{center}
\textbf{\text{\Large  Anexo 2 - Fluxo de funcionamento proximaLinha}}
\begin{tikzpicture}[>=stealth',shorten >=1pt,node distance=3cm,on grid,initial/.style    ={}]
  \node[state]          (P)                        {$pxL$};
  \node[state]          (B) [ above =of P]    {$aleM$};
  \node[state]          (C) [right =of B]    {$marD$};
    \node[state]          (M) [above right =of C]    {$marE$};
  \node[state]          (F) [above left   =of P]    {$iIlha$};
  \node[state]          (G) [right =of P]    {$velpxL$};
  \node[state]          (R) [below =of G]    {$rRand$};
  \node[state]          (H) [below left =of P]    {$nLz$};
\tikzset{mystyle/.style={<->,double=orange}} 
\tikzset{every node/.style={fill=white}} 
\path (P)     edge [mystyle]    node   {$1$} (B)
      (B)     edge [mystyle]    node   {$2$} (M)
      (B)     edge [mystyle]    node   {$3$} (C)  
      (P)     edge [mystyle]    node   {$4$} (F)
      (P)     edge [mystyle]    node   {$5$} (G)
      (G)     edge [mystyle]    node   {$6$} (M)
      (G)     edge [mystyle]    node   {$7$} (C)
      (G)     edge [mystyle]    node   {$8$} (R)
      (P)     edge [mystyle]    node   {$9$} (H);
\end{tikzpicture}
\end{center}
\textbf{\text{  proximaLinha - Legenda }}
\begin{enumerate}
\item[-]{pxL - proximaLinha}
\item[-]{aleM - aleatorizaMargem}
\item[-]{marE - margemEsquerda}
\item[-]{marD - margemDireita}
\item[-]{iIlha - InsereIlha}
\item[-]{velpxL - velocidadeProximaLinha}
\item[-]{rRand - realRandomico}
\item[-]{nLz - normaliza}
\end{enumerate}

\end{document}